\documentclass[12pt, leqno]{article}
\usepackage{amsfonts}
\usepackage{amsmath}
\usepackage{fancyhdr}
\usepackage{hyperref}
\usepackage{tikz}
\usepackage{pgfplots}
\usepackage{listings}

\newcommand{\fl}{\operatorname{fl}}
\newcommand{\macheps}{\epsilon_{\mathrm{mach}}}
\newcommand{\bbR}{\mathbb{R}}
\newcommand{\bbC}{\mathbb{C}}

\newcommand{\hdr}[2]{
  \pagestyle{fancy}
  \lhead{Bindel, Spring 2015}
  \rhead{Numerical Analysis (CS 4220)}
  \fancyfoot{}
  \begin{center}
    {\large{\bf #1}} \\
    Due: #2
  \end{center}
  \lstset{language=matlab,columns=flexible}
}

\newcommand{\phdr}[1]{
  \pagestyle{fancy}
  \lhead{Bindel, Spring 2015}
  \rhead{Numerical Analysis (CS 4220)}
  \fancyfoot{}
  \begin{center}
    {\large{\bf #1}}
  \end{center}
  \lstset{language=matlab,columns=flexible}
}


\begin{document}
\hdr{2015-03-09}

\section*{Splittings and Sweeps}

In the last lecture, we discussed stationary iterative methods using
matrix notation.  Specifically, we said a stationary method for
the equation $Ax = b$ is associated with a {\em splitting} $A = M-N$,
and convergence (or lack thereof) is associated with the iteration
matrix $R = M^{-1} N$.  This is the right linear algebraic framework
for discussing convergence of these methods, but it is not the way
they are usually programmed.  The connection between a matrix
splitting and a ``sweep'' of a stationary iteration like Gauss-Seidel
or Jacobi iteration is not always immediately obvious, and so it is
probably worth spending a moment or two explaining in more detail.

For the sake of concreteness, let's consider a standard model problem:
a discretization of a Poisson equation on a line.  That is, we
approximate
\[
  -\frac{d^2 u}{dx^2} = f, \quad u(0) = u(1) = 0
\]
using the second-order finite difference approximation
\[
  \frac{d^2 u}{dx^2} \approx \frac{u(x-h)-2u(x)+u(x+h)}{h^2}
\]
where $h$ is a small step size.  We discretize the problem by meshing
$[0,1]$ with evenly spaced points $x_j = jh$ for $j = 0$
to $N+1$ where $h = 1/(N+1)$, then apply this approximation at each
point.  This procedure yields the equations
\[
  -u_{j-1}+2u_j-u_{j+1} = h^2 f_j, \quad j = 1, \ldots, N
\]
or, in matrix notation
\[
  Tu = h^2 f
\]
where $u$ and $f$ are vectors in $\bbR^N$ representing the sampled
(approximate) solution and the sampled forcing function.  The matrix
$T$ is a frequently-recurring model matrix, the tridiagonal
\[
T =
\begin{bmatrix}
   2 & -1 \\
  -1 &  2 & -1 \\
     & -1 &  2 & -1 \\
     &    & \ddots & \ddots & \ddots \\
     &    &        & -1 & 2 & -1 \\
     &    &        &    & -1 & 2
\end{bmatrix}.
\]

Suppose we forgot about how cheap Gaussian elimination is for
tridiagonal matrices.  How might we solve this system of equations?
A natural thought is that we could make an initial guess at the
solution, then refine the solution by ``sweeping'' over each node $j$
and adjusting the value at that node ($u_j$) to be consistent with
the values at neighboring nodes.  In one sweep, we might compute
a new set of values $u^{\mathrm{new}}$ from the old values $u^{\mathrm{old}}$:
\begin{lstlisting}
  for j = 1:N
    unew(j) = (h^2*f(j) + uold(j-1) + uold(j+1))/2;
  end
\end{lstlisting}
or we might update the values for each node in turn, using the most
recent estimate for each update, i.e.
\begin{lstlisting}
  for j = 1:N
    u(j) = (h^2*f(j) + u(j-1) + u(j+2))/2;
  end
\end{lstlisting}
These are, respectively, a step of {\em Jacobi} iteration and a step
of {\em Gauss-Seidel} iteration, which are two standard stationary
methods.

How should we relate the ``sweep'' picture to a matrix splitting?
The update equation from step $k$ to step $k+1$ in Jacobi is
\[
  -u_{j-1}^{(k)}+2u_j^{(k+1)}-u_{j+1}^{(k)} = h^2 f_j, 
\]
while the Gauss-Seidel update is
\[
  -u_{j-1}^{(k+1)}+2u_j^{(k+1)}-u_{j+1}^{(k)} = h^2 f_j.
\]
In terms of splittings, this means that Jacobi corresponds to
taking $M$ to be the diagonal part of the matrix,
\[
M =
\begin{bmatrix}
   2 &    \\
     &  2 &    \\
     &    &  2 &    \\
     &    &        & \ddots &  \\
     &    &        &    & 2 &  \\
     &    &        &    &   & 2
\end{bmatrix}, ~~
N =
\begin{bmatrix}
   0 &  1 \\
   1 &  0 &  1 \\
     &  1 &  0 & 1 \\
     &    & \ddots & \ddots & \ddots \\
     &    &        & 1 & 0 & 1 \\
     &    &        &   & 1 & 0
\end{bmatrix},
\]
while Gauss-Seidel corresponds to taking $M$ to be the lower triangle
of the matrix,
\[
M =
\begin{bmatrix}
   2 &  \\
  -1 &  2 &  \\
     & -1 &  2 &  \\
     &    & \ddots & \ddots &  \\
     &    &        & -1 & 2 &  \\
     &    &        &    & -1 & 2
\end{bmatrix}, ~~
N =
\begin{bmatrix}
   0 &  1 \\
     &  0 &  1 \\
     &    &  0 & 1 \\
     &    &    & \ddots & \ddots \\
     &    &        &  & 0 & 1 \\
     &    &        &    &   & 0
\end{bmatrix}.
\]

The point of this exercise is that {\em programming} stationary
iterative methods and {\em analyzing} the same methods may lead
naturally to different ways of thinking about the iterations.
It's worthwhile practicing mapping back and forth between these
two modes of thought.

\section*{Linear Solves and Quadratic Minimization}

Last time, we briefly described an argument that Jacobi iteration
converges for strictly row diagonally dominant matrices.  We now
discuss an argument that Gauss-Seidel converges (or at least part of
such an argument).  In the process, we will see a useful way of
reformulating the solution of symmetric positive definite linear
systems that will prepare us for our upcoming discussion of conjugate
gradient methods.

Let $A$ be a symmetric positive definite matrix, and consider
the ``energy'' function
\[
  \phi(x) = \frac{1}{2} x^T A x - x^T b.
\]
The stationary point for this function is the point at which
the derivative in any direction is zero.  That is, for any
direction vector $u$,
\begin{align*}
  0
  &=\left. \frac{d}{d\epsilon} \right|_{\epsilon = 0} \phi(x+\epsilon u) \\
  &=\frac{1}{2} u^T A x + \frac{1}{2} x^T A u -  u^T b \\
  &=u^T (Ax-b)
\end{align*}
Except in pathological instances, a directional derivative can be
written as the dot product of a direction vector and a gradient;
in this case, we have
\[
  \nabla \phi = Ax-b.
\]
Hence, minimizing $\phi$ is equivalent to solving $Ax = b$~\footnote{%
If you are unconvinced that this is a minimum, work through the
algebra to show that $\phi(A^{-1} b + w) = \frac{1}{2} w^T A w$ for any $w$.}.

Now that we have a characterization of the solution of $Ax = b$
in terms of an optimization problem, what can we do with it?
One simple approach is to think of a sweep through all the unknowns,
adjusting each variable in term to minimize the energy; that is,
we compute a correction $\Delta x_j$ to node $j$ such that
\[
  \Delta x_j = \operatorname{argmin}_z \phi(x+z e_j)
\]
Note that
\[
  \frac{d}{dz} \phi(x+z e_j) = e_j^T (A(x+ze_j)-b),
\]
and the update $x_j := x_j + \Delta x_j$ is equivalent to choosing
a new $x_j$ to set this derivative equal to zero.  But this is
exactly what the Gauss-Seidel update does!  Hence, we can see
Gauss-Seidel in two different ways: as a stationary method for solving
a linear system, or as an optimization method that constantly makes
progress toward a solution that minimizes the energy~\footnote{%
Later in the class, we'll see this as coordinate-descent with
exact line search.}.
The latter perspective can be turned (with a little work) into
a convergence proof for Gauss-Seidel on positive-definite linear systems.

\section*{Extrapolation: A Hint of Things to Come}

Stationary iterations are simple.  Methods like Jacobi or Gauss-Seidel
are easy to program, and it's (relatively) easy to analyze their
convergence.  But these methods are also often slow.  We'll talk next
time about more powerful {\em Krylov subspace} methods that use
stationary iterations as a building block.

There are many ways to motivate Krylov subspace methods.  We'll end
this lecture by picking one motivating idea that extends beyond
the land of linear solvers and into other applications as well.
The key to this idea is the observation that the error in our
iteration follows a simple pattern:
\[
  x^{(k)}-x = e^{(k)} = R^k e^{(0)}, \quad R = M^{-1} N.
\]
For large $k$, the behavior of the error is dominated by the largest
eigenvalue and the associated eigenvector\footnote{%
If you don't understand this now, you will when we talk about the
power method in a week or so!}, i.e.
\[
  e^{(k+1)} \approx \lambda_1 e^{(k)}.
\]
Note that this means
\[
  x^{(k)}-x^{(k+1)} = e^{(k)}-e^{(k+1)} \approx (1-\lambda_1) e^{(k)}.
\]
If we have an estimate for $\lambda_1$, we can write
\[
  x = x^{(k)} - e^{(k)} \approx
  x^{(k)}-\frac{x^{(k)}-x^{(k+1)}}{1-\lambda_1}.
\]
That is, we might hope to get a better estimate of $x$ than is
provided by $x^{(k)}$ or $x^{(k+1)}$ individually by taking an
appropriate linear combination of $x^{(k)}$ and $x^{(k+1)}$.  This
idea generalizes: if we have a sequence of approximations
$x^{(0)}, \ldots, x^{(k)}$, why not ask for the ``best'' approximation
that can be written as a linear combination of the $x^{(j)}$?
This is the notion underlying methods such as the conjugate iteration
iteration, which will be the subject (at least in part) of our next
lecture.

\end{document}
